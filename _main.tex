% Options for packages loaded elsewhere
\PassOptionsToPackage{unicode}{hyperref}
\PassOptionsToPackage{hyphens}{url}
%
\documentclass[
  openany]{book}
\usepackage{lmodern}
\usepackage{amssymb,amsmath}
\usepackage{ifxetex,ifluatex}
\ifnum 0\ifxetex 1\fi\ifluatex 1\fi=0 % if pdftex
  \usepackage[T1]{fontenc}
  \usepackage[utf8]{inputenc}
  \usepackage{textcomp} % provide euro and other symbols
\else % if luatex or xetex
  \usepackage{unicode-math}
  \defaultfontfeatures{Scale=MatchLowercase}
  \defaultfontfeatures[\rmfamily]{Ligatures=TeX,Scale=1}
\fi
% Use upquote if available, for straight quotes in verbatim environments
\IfFileExists{upquote.sty}{\usepackage{upquote}}{}
\IfFileExists{microtype.sty}{% use microtype if available
  \usepackage[]{microtype}
  \UseMicrotypeSet[protrusion]{basicmath} % disable protrusion for tt fonts
}{}
\makeatletter
\@ifundefined{KOMAClassName}{% if non-KOMA class
  \IfFileExists{parskip.sty}{%
    \usepackage{parskip}
  }{% else
    \setlength{\parindent}{0pt}
    \setlength{\parskip}{6pt plus 2pt minus 1pt}}
}{% if KOMA class
  \KOMAoptions{parskip=half}}
\makeatother
\usepackage{xcolor}
\IfFileExists{xurl.sty}{\usepackage{xurl}}{} % add URL line breaks if available
\IfFileExists{bookmark.sty}{\usepackage{bookmark}}{\usepackage{hyperref}}
\hypersetup{
  pdftitle={Transurfing Redux},
  pdfauthor={Caroline C.},
  hidelinks,
  pdfcreator={LaTeX via pandoc}}
\urlstyle{same} % disable monospaced font for URLs
\usepackage{longtable,booktabs}
% Correct order of tables after \paragraph or \subparagraph
\usepackage{etoolbox}
\makeatletter
\patchcmd\longtable{\par}{\if@noskipsec\mbox{}\fi\par}{}{}
\makeatother
% Allow footnotes in longtable head/foot
\IfFileExists{footnotehyper.sty}{\usepackage{footnotehyper}}{\usepackage{footnote}}
\makesavenoteenv{longtable}
\usepackage{graphicx,grffile}
\makeatletter
\def\maxwidth{\ifdim\Gin@nat@width>\linewidth\linewidth\else\Gin@nat@width\fi}
\def\maxheight{\ifdim\Gin@nat@height>\textheight\textheight\else\Gin@nat@height\fi}
\makeatother
% Scale images if necessary, so that they will not overflow the page
% margins by default, and it is still possible to overwrite the defaults
% using explicit options in \includegraphics[width, height, ...]{}
\setkeys{Gin}{width=\maxwidth,height=\maxheight,keepaspectratio}
% Set default figure placement to htbp
\makeatletter
\def\fps@figure{htbp}
\makeatother
\setlength{\emergencystretch}{3em} % prevent overfull lines
\providecommand{\tightlist}{%
  \setlength{\itemsep}{0pt}\setlength{\parskip}{0pt}}
\setcounter{secnumdepth}{5}

\title{Transurfing Redux}
\author{Caroline C.}
\date{}

\begin{document}
\maketitle

{
\setcounter{tocdepth}{1}
\tableofcontents
}
\hypertarget{transurfing---notes-quotes-and-reflections}{%
\chapter{Transurfing - Notes, Quotes, and Reflections}\label{transurfing---notes-quotes-and-reflections}}

I'm writing this mini bookdown to help myself understand \href{https://www.amazon.com/Reality-transurfing-Steps-Vadim-Zeland/dp/1532814658}{\emph{Transurfing}} and \href{https://www.amazon.com/Transurfing-78-Days-Practical-Creating/dp/5957334715/ref=pd_lpo_1?pd_rd_i=5957334715\&psc=1}{\emph{Transurfing in 78 Days}} by \href{https://www.amazon.com/Vadim-Zeland/e/B00J0SESMY/ref=dp_byline_cont_pop_book_1}{Vadim Zeland}.

\hypertarget{pendulums}{%
\chapter{Pendulums}\label{pendulums}}

When a group of people think in the same way, an energetic structure appears. The energetic structure is an \textbf{energy pendulum}. The group's thought energy becomes identical. The parameters of what they think, that is, how much they are capable of pushing the boundaries of what they think, is defined by the pendulums. The energy unites into a single current; the \emph{energy pendulum} eventually begins to live its own life, and the group that created it is now subjugated by the pendulum's laws.

The structure is referred to as a pendulum because when more people attune to the thought energy that corresponds to the structure, the more powerfully the energy sways in a concentrated motion, like a pendulum gaining momentum.

A pendulum will cease to exist when people stop contributing to its thought energy. Examples of defunct pendulums include ancient pagan religions, stone tools, jump drives, beta and VHS tapes, etc.

The most important thing is to be able to recognize a pendulum, and unless you stand to benefit from its sway, you must avoid getting swept up by its energy.

One identifying factor of a destructive pendulum is that it competes with similar pendulums in its battle for adherents. A pendulum has only one goal: to attract as many supporters as possible so that it can feed off their energy. The more agressive a pendulum behaves, the more dangerous it is to the life of an individual.

Imagine a group of people criticizing the government. They obviously don't agree with the government pendulum. However, because they feel passionately in their discontent, their thought energy matches the frequency of the government pendulum and therefore feeds it. It doesn't matter whether you are pro or anti-pendulum --- your energy still feeds it no matter which side you are on.

The list of strings the pendulum uses to control its puppets is endless: justice, pride, ambition, honor, love, hate, greed, generosity, curiousity, interest, hunger, neediness, etc.

The harder you try to avoid something, the more likely it is that you will encounter it. Just thinking negative thoughts and fertilizing them with emotion is enough to shift you onto undesired life lines.

The only way to eliminate unwanted things from your life is to break free from the influence of the pendulums that you have glommed on to. From then on, you must avoid responding to destructive pendulums and avoid getting pulled into their sway. Basically, just do your best to think about something else.

Everything you see and experience should be accepted as if it were were all paintings on display, regardless of whether or not you like them. Recognize that every pendulum has a right to exist, and that you have the right to leave every pendulum alone (and therefore remain free of its influence). Any opposition to the pendulum supplies it with energy. \emph{To eliminate something from you life, simply ignore it.}

Of course, it isn't always easy to ignore certain pendulums. If you disagree with your boss, for example, fighting with them amounts to fighting with the boss-pendulum. You can, however, act as if you are willing to play along but always be aware that you are only pretending. When you behave in a calm, unperturbed manner, the pendulum cannot hook onto you and will sway right by you.

When you habitually react negatively to unpleasant circumstances, you trigger a pendulum's mechanism for capturing thought energy. This habit will fade once you decide to play your own game in which you deliberately substitute negative emotion with positive emotion: confidence for fear, enthusiasm for gloom, indifference for resentment, joy for iritation. When you radiate your thought energy at a different frequency than the pendulum's resonance frequency, you are in dissonance with the pendulum. The pendulum can't do anything but leave you in peace.

When you encounter someone unpleasant/an unpleasant pendulum, imagine that person-pendulum feeling content and happy. Distract yourself from the unpleasantness. A person-pendulum does not sway randomly. It seeks only to restore its balance. The energy of your thoughts set to a certain positive frequency will restore the person-pendulum's balance, instantly substituting aggression with goodwill.

When a pendulum comes to rest, the energy comes to you! You become stronger. The next time you use the transurfing technique, it will be easier.

The secret of genius lies in remaining free from the influence of pendulums. The thought energy of most people is held at certain frequencies by pendulums. But the thought energy of a genius is free to attune itself to frequencies of her own choosing and she can therefore move into unexplored areas of the information field.

To avoid getting caught in a pendulum's snare, keep your problems in perspective while remaining aware of the pendulum's intent to trick you into jumping onto its swing. Act as you normally would but remain observant. Stay out of the game. Don't be emotionally attached to any outcomes. Even if your initial reaction is fear, confusion, resentment, etc., it is important to change that reaction as quickly as possible. Transform your negative reaction into its exact opposite.

Your subconscious is directly linked to the information field where all solutions already exist. Relax. Let go. Quiet the chatter of the mind and contemplate the nature of emptiness. This exercise will develop your ability to obtain knowledge through your intuition. Practice it until it becomes a habit.

Once you learn to avoid destructive pendulums, you are free to attune yourself to pendulums of your choosing.

\end{document}
